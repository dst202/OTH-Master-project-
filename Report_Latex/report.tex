% arara: pdflatex: { shell: yes } until !found('log', '\\(?(R|r)e\\)?run (to get|LaTeX)')
\documentclass[fontsize=12pt,
parskip=half,	% Abstände statt Einrückungen bei Absätzen
department=FakM,  % Farbanpassungen
twoside, % Spart Papier und erhöht die Lesbarkeit
DIV=15,BCOR=10mm, % Seitenlayout wie bei Koma-Script
svgnames,table,hyperref, % Optionen für xcolor
bookmarks, raiselinks, pageanchor, hyperindex, colorlinks, hidelinks, % Optionen für hyperref
]{OTHRreprt}
\usepackage[utf8x]{inputenc}
\usepackage[english,ngerman]{babel}


%\usepackage{parskip}			% Einstellungen fuer Absaetze: Abstand statt Einrueckung

%\usepackage[a4paper,		% Papierformat A4
%left=3.5cm,				% linker Rand
%right=2.5cm,				% rechter Rand
%top=2.5cm,				% oberer Rand
%bottom=2.0cm,			% unter Rand
%marginparsep=5mm,		% Abstand der Randnotizen
%marginparwidth=10mm, 	% Breite der Randnotizen
%headheight=3.5mm,		% Hoehe der Kopfzeile
%headsep=0.5cm,			% Abstand der Kopfzeile
%footskip=1.5cm,			% Abstand der Fusszeile
%includeheadfoot]{geometry}

%\usepackage{amsmath}				% Pakete fuer den Mathematikmodus
%\usepackage{amssymb}
%\usepackage[intlimits]{empheq}
%\usepackage{dsfont}
%\usepackage{courier}
%\usepackage{graphicx}
%\usepackage[sc]{mathpazo}		% Schriftart Palatino fuer Haupttext und Mathematikmodus
\usepackage{pifont}				% zusaetzliche Symbole
\usepackage{acronym} % Abkürzungsverzeichnis


\usepackage[format=hang,			% Einstellung fuer Bildunterschriften
font={footnotesize},
labelfont={bf},
margin=1cm,
aboveskip=5pt,
position=bottom]{caption}

\usepackage{booktabs} % Hübschere Tabellen
%\usepackage{graphicx}							% Einbinden von Graphiken
%\usepackage[svgnames,table,hyperref]{xcolor} 		% Verwendung von Farben
%\usepackage{tikz}								% Erstellen von Grafiken
%\usetikzlibrary{positioning,arrows,plotmarks} 	% TikZ-Bibliotheken

\usepackage[autostyle=true,german=quotes]{csquotes}	% Zur Nutzung von deutschen Anführungszeichen, innerhalb des Textes mit dem Befehl \enquote vorgehen
\usepackage[bottom]{footmisc}
\usepackage[gen]{eurosym}				% Eurozeichen einfügen
\usepackage{chngpage}					
%\usepackage{lscape}						% Nützlich, falls querformatierte 	Seiten gewünscht sind
%\usepackage{pdflscape}					% Zum exportieren der Landscapes in PDF-Dateien
\usepackage{lipsum}						% Dummy-Text generieren

\usepackage{fancyhdr}					% Konfiguration von Kopf- und Fusszeilen
\pagestyle{fancy}						% Seitenstil 'fancy'
\fancyhf{}								% vorhandene Einstellungen loeschen
\setlength{\headwidth}{\textwidth}		% Kopf- und Fusszeile so breit wie der Haupttext
\fancyfoot[C]{\thepage} 					% Festlegung des Seitenstils: Seitenzahlen in der Fusszeile rechts
\fancyhead[R]{\leftmark}					% Kapitelnr. und -Bezeichnung in der Fusszeile links
\renewcommand{\chaptermark}[1]{			% Definition der Ausgabe des Kapitels
  \markboth{\thechapter. #1}{}}
\renewcommand{\headrulewidth}{0.5pt}		% Trennlinie zwischen Kopfzeile und Haupttext
%\renewcommand{\footrulewidth}{0.5pt}	% Trennlinie zwischen Haupttext und Fusszeile
\fancypagestyle{plain}{					% Anpassung des Seitenstils 'plain' bei Beginn neuer Kapitel
  \fancyhf{}								% Vorbelegung loeschen
  \fancyfoot[C]{\thepage}				% Seitenzeilen in der Fusszeile mittig
  \fancyhead[R]{\leftmark}				% Kapitelnr. und -Bezeichnung in der Fusszeile links
}

\documenttype{Project Report}
\title{USB to Fiber Optic Adapter using SFP module}
\author{Devi Surya Teja Chilukuri}
\studentid{3330180}
\department{Faculty of Applied Natural Sciences and Cultural Studies}
\studyprogramme{Master in Electrical and Microsystems Engineering}
\startingdate{1.\,October 2022}
\closingdate{13.\,March 2023}
\firstadvisor{Prof. Dr. Thomas Fuhrmann}
% Hiermit trägt pdflatex die PDF-Metadaten des erzeugten Dokuments ein:
\makeatletter

\hypersetup{pdftitle={\,USB to Fiber Optic Adapter using SFP module},%
	pdfauthor={\,Devi Surya Teja Chilukuri},%
	%pdfsubject={Optionaler Untertitel / englischer Titel},%
	%pdfkeywords={Optionale Schlüsselwörter}
	}
\makeatother

\begin{document}
	\maketitle
	\cleardoublepage
	\pagenumbering{roman}
	\begin{abstract}
	\section*{Abstract}
	\begin{quote}
	This report presents the development of a USB to SFP fiber optic adapter using 1 gigabit SFP modules and the Asix AX88772b IC. The aim of this project was to create a cost-effective and reliable solution for converting USB data to optical signals. The adapter was designed to operate at a data rate of 1 Gbps and supports full duplex communication.The hardware design of the adapter includes a USB connector, a power regulator, the Asix AX88772b IC, and the 1 gigabit SFP module. The adapter was tested using standard optical testing equipment to evaluate its performance in terms of signal integrity and transmission distance. The design and implementation of this adapter offer several advantages, such as the ability to use off-the-shelf components, cost-effectiveness, and reliability. This adapter can be used in various applications, such as in data centers, communication networks, and industrial environments. The adapter's ability to convert USB data to optical signals offers a solution for extending the distance of USB connections beyond the limitations of copper cables.In conclusion, this project demonstrates the feasibility of using 1 gigabit SFP modules and the Asix AX88772b IC to develop a USB to SFP fiber optic adapter that offers a cost-effective, reliable, and high-speed data transmission solution
	\end{quote}
	\end{abstract}
	\cleardoublepage
	{
		\hypersetup{linkcolor=black}	% Färbung von Links innerhalb des Inhaltsverzeichnisses auf schwarz setzen
		\tableofcontents
	}
	
	\cleardoublepage		


	\pagenumbering{arabic}
		
	%\include{Kapitel1}	% Die Kapitel als seperate .tex Datei im Ordner abspeichern. Dort dann Befehle wie \chapter{} und \section{} sowie \subsection{} verwenden (Keine neue "documentclass!")
	%\include{Kapitel2}
	
	%%% Die folgenden Zeilen dienen nur zur Veranschaulichung des Textlayouts, sie sollten später gelöscht werden!
	
	\chapter{Introduction}
	\begin{quote}
	The rapid development of data communication technology has led to the increasing demand for high-speed data transmission solutions. Fiber optic technology offers a reliable and high-speed data transmission option, but it has typically been more expensive and difficult to implement than traditional copper cable solutions. As a result, there is a need for cost-effective and reliable fiber optic solutions that can be easily implemented.

One solution to this problem is the use of USB to SFP fiber optic adapters. These adapters provide a way to convert USB data to optical signals, which can be transmitted over long distances without experiencing significant signal loss. The availability of 1 gigabit SFP modules and the Asix AX88772b IC has made it possible to develop a cost-effective and reliable USB to SFP fiber optic adapter.

The aim of this project is to develop a USB to SFP fiber optic adapter using 1 gigabit SFP modules and the Asix AX88772b IC. The adapter is designed to provide a cost-effective and reliable solution for converting USB data to optical signals. The adapter is expected to offer a high-speed data transmission solution that can operate at a data rate of 1 Gbps and support full duplex communication.

The hardware design of the adapter includes a USB connector, a power regulator, the Asix AX88772b IC, and the 1 gigabit SFP module. The software for the adapter was developed using the libusb library, which provides a user-friendly interface for configuring the adapter.The design and implementation of this adapter offer several advantages, such as the ability to use off-the-shelf components, cost-effectiveness, and reliability. This adapter can be used in various applications, such as in home servers, communication networks, and industrial environments. The adapter's ability to convert USB data to optical signals offers a solution for extending the distance of USB connections beyond the limitations of copper cables.

In conclusion, the development of a USB to SFP fiber optic adapter using 1 gigabit SFP modules and the Asix AX88772b IC offers a cost-effective, reliable, and high-speed data transmission solution for converting USB data to optical signals. This project demonstrates the feasibility of using off-the-shelf components to develop a USB to SFP fiber optic adapter suitable for various applications in the field of telecommunications and data communication
	\end{quote}
	
	\section{List of Components used}
	
	\subsection{SFP Transciever Module}
	\begin{quote}
	SFP Transciever modules have both optical transiever and reciver built into them on one side and SFP plug on other side.The SFP moules used here is DELOCK 86027 Mini GBIC, 100Base-FX.This is a SFP module that supports a data rate of 100 Mbps. It provides an interface for connecting the adapter to the optical network. The DELOCK 86027 is designed to be hot-swappable, which means it can be inserted and removed from the adapter while the power is on. The module is also equipped with an LC connector, which is a small, square-shaped connector commonly used in fiber optic networks. The 100Base-FX standard is used to transmit data over fiber optic cables and has a maximum distance of 2 km.
	\end{quote}
	
	\subsection{USB to Ethernet controller}
	\begin{quote}
	USB to Ethernet controllers are specifically designed to enable USB devices to connect to Ethernet networks. They provide a way to convert USB data signals into Ethernet packets that can be transmitted over an Ethernet network.The controller we used here is ASIX AX88772B.It is a high-performance and low-power USB 2.0 to 10/100/1000 Gigabit Ethernet controller designed for a wide range of applications. It is a highly integrated controller that features a built-in 10/100/1000 Ethernet MAC and PHY, and supports full-duplex with flow control and half-duplex with backpressure operation.It interfaces with the host system via a USB 2.0 Hi-Speed device port, which supports high-speed data transfer rates of up to 480Mbps

In summary, the ASIX AX88772B USB to Ethernet controller is a high-performance, low-power, and versatile device that is widely used in a variety of applications. It provides a cost-effective and easy-to-use solution for adding Ethernet connectivity to USB devices and systems.
	\end{quote}
	\subsection{Voltage Regulator}
	\begin{quote}
	Voltage regulator is necessary here to provide power to USB Ethernet  controller and SFP module.The Voltage regulator used here is LM 117 in SOT-23 package..The LM117 is a three-terminal positive voltage regulator that provides a fixed output voltage of 1.2V to 37V with up to 1.5A of output current. It is a linear regulator that operates by adjusting the voltage drop across a pass transistor to maintain a stable output voltage, and is designed to operate with a wide range of input voltages (up to 40V).

	The LM117 is widely used in a variety of applications that require a stable, regulated output voltage, including power supplies, battery chargers, and voltage references. It features built-in thermal overload protection, short circuit protection, and safe area protection, which help to ensure reliable operation even under harsh operating conditions.
	\end{quote}	
	
	\subsection{Serial EEPROM}
	\begin{quote}
	The USB Ethernet controller operates  by default in copper mode.We need an EEPROM to put the USB Ethernet controller in Fiber mode instead of Copper mode.Microchip 93C66C in SOIC package  is used here. Microchip 93C66C is a low-power, 4K-bit serial electrically erasable programmable read-only memory (EEPROM) that operates over a wide voltage range. It features a three-wire serial interface that supports high-speed data transfer rates of up to 3.4Mbps, and supports both byte and page write operations.

One of the key advantages of the 93C66C EEPROM is its small form factor and low power consumption. It is available in a variety of small packages, including 8-pin PDIP, SOIC, and TSSOP, making it easy to integrate into a wide range of systems and devices. The low power consumption of the device makes it an ideal choice for battery-powered applications, and its small form factor allows it to be used in applications with limited space.
	\end{quote}
	
	\subsection{P-Channel MOSFET}
	\begin{quote}
	In this project, the P-Channel Mosfet BS2123LT1G is used to control the LED indicators for the Ethernet connection status and activity.he Mosfet BS2123LT1G is a P-channel MOSFET transistor designed for low voltage applications. It has a maximum voltage of -20V and a maximum current of -400mA. The transistor has a low on-resistance, allowing for efficient switching of signals. The BS2123LT1G has a compact and rugged design, making it suitable for use in a variety of applications, including power management and motor control
	 The transistor is connected to the microcontroller, which sends signals to the gate of the Mosfet to switch the LEDs on and off. The low on-resistance of the Mosfet allows for efficient switching of the LED signals, while the compact design of the transistor makes it suitable for use in the small form factor USB-to-SFP adapter.
	 \end{quote}
	 
	\subsection{Fiber Optic Cable}
	\begin{quote}
	The SFP modules on both adapters are connected by Fiber Optic cable.In this project, the Digitus D2533024 LWL Multimode Patchkabel is used to connect the Delock 86027 Mini GBIC 100Base-FX module on both ends of adapters. The high-quality fiber and connectors in the cable ensure a reliable and high-speed connection between the SFP module and the fiber optic network, enabling high-performance data transfer over the network.
	The Digitus D2533024 LWL Multimode Patchkabel is a high-quality fiber optic patch cable that is used to connect optical devices over a short distance. It has a length of 2 meters and LC connectors on both ends. The cable is made of high-quality multimode fiber that provides a high data transfer rate and a low attenuation over the cable length.
	\end{quote} 
	
	\subsection{Micro USB connector}
	\begin{quote}
	This is a small and compact USB connector that is commonly used in mobile devices and other small electronics. In this project, it is used to connect the USB-to-SFP adapter to a USB port on a computer or other USB host device.
		\end{quote} 
	\end{quote} 

	
	\section{Circuit and PCB Design}
	\begin{quote}
	The main circuit design is offered by ASIX electronics who designed AX88772B.The entire circuit is converted to schematic and a PCB designed in EAGLE considering smallest possible distance between components because of high frequency of signals.The top and bottom of PCB also have grounding plane to reduce noise in signals.The PCB is designed in dual side configuration.
	\end{quote}	
	
	\section{EEPROM Programming}
	\begin{quote}
	
	
	
	\end{{quote}	
	
	
	\chapter{Musterhauptteil}
	
	
	% Literaturverzeichnis in das Inhaltsverzeichnis einfügen
	\addcontentsline{toc}{chapter}{Literaturverzeichnis}
	
	% Style für die Bibliothek festlegen
  	\bibliographystyle{plain}
  	
  	% Einfügen des Literaturverzeichnisses in das Dokument
	%\bibliography{Musterpfad/Zum/Literaturverzeichnis}
	
	\newpage
		
	\thispagestyle{empty}	
	\addcontentsline{toc}{chapter}{Abkürzungsverzeichnis}\label{Sec:Abkuerzungen}
	\chapter*{Abkürzungsverzeichnis}
	\begin{acronym}[OTH R]
	 \acro{OTH R}{Ostbayerische Technische Hochschule Regensburg}
	 \acro{OTH}{Ostbayerische Technische Hochschule}
	 \acro{GCC}{Der Compiler namens gcc}
	 \acro{I2C}[I²C]{Inter-Integrated Circuit}
	\end{acronym}
	
	% Anhang
	\addcontentsline{toc}{chapter}{Abbildungsverzeichnis}
	\listoffigures
	
	\addcontentsline{toc}{chapter}{Tabellenverzeichnis}
	\listoftables
	
	%\addcontentsline{toc}{chapter}{Listingsverzeichnis} % für Quellcode
	%\lstlistoflistings 
	
	%\addcontentsline{toc}{chapter}{Digitaler Anhang}	% für digitalen Anhang, falls nötig
	%\include{anhang}
	\cleardoublepage
	\thispagestyle{empty}
	\makedeclaration

	\cleardoublepage
\makedeclaration

\end{document}
